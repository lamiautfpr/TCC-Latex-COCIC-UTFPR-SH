%%%% ELEMENTOS PRÉ-TEXTUAIS
%%
%% Parte que antecede o texto com informações que ajudam na identificação e na
%% utilização do trabalho.
%%
%% Observações:
%% - {Arg} argumento obrigatório de ambiente ou comando;
%% - [Arg] argumento opcional de ambiente ou comando.

%% Folha de rosto
%% Contém os elementos essenciais à identificação do trabalho, além de uma
%% licença Creative Commons (https://creativecommons.org/choose/).
%% Ambiente {TitlePage*}: aplica caixa alta no título em idioma secundário.
\begin{TitlePage}%% Argumentos (2):
[BY-NC-SA]%% Tipo de licença (BY, BY-SA, BY-ND, BY-NC, BY-NC-SA ou BY-NC-ND)
% [Texto da licença]%% Substitui o texto padrão para cada tipo de licença
%%%% Descrição do trabalho (padrão; alterar se necessário)
\DocumentTypeName\ apresentad\ifbool{Graduate}{a}{o} como requisito para obtenção do título de \StudentsTitlesList\ em \CourseName\ da \UTFPRName\ (\intl*{UTFPR}).
%%%% Ficha catalográfica (somente para Teses e Dissertações em catálogo físico)
%%%% [Arg-1]: local (pasta) do PDF (./Pre-Textual/ por padrão).
%%%% {Arg-2}: nome do PDF (em ./Pre-Textual/; modelo em ./Pre-Textual/Extras/).
% \IndexCardPDF{doc-index-card.pdf}
\end{TitlePage}

%% Errata (elemento opcional)
%% Lista dos erros ocorridos no texto, seguidos das devidas correções.
%% Ambiente {Errata*}: insere a autorreferência do documento.
% \begin{Errata}%[Título Alternativo]%% Substitui o título padrão
% %%%% Formato (com \midrule entre linhas): Página(s) & Onde se lê & Leia-se \\
% \labelcpageref{err:chpt-1,err:chpt-2,err:chpt-3,err:chpt-4,err:chpt-5,err:chpt-6} &
% capítulo{(s)}                                                                     &
% seção{(ões)} primária{(s)}                                                        \\
% \midrule%
% \pageref{err:sect}           &
% seção{(ões)}                 &
% seção{(ões)} secundária{(s)} \\
% \midrule%
% \pageref{err:ssect}         &
% subseção{(ões)}             &
% seção{(ões)} terciária{(s)} \\
% \end{Errata}

%% Folha de aprovação
%% Contém os elementos essenciais à aprovação do trabalho (sem as assinaturas).
%%%% Opção 1: gerada por meio do pacote lamia-tcc-utfpr-sh.
%%%% Ambiente {ApprovalPage*}: insere a titulação após o nome do membro.
\begin{ApprovalPage}%% Argumentos (4):
% [brazilian]%% Idioma original ou primário (brazilian ou english)
% {12 de Novembro de 2023}%% Data de aprovação (dia, mês por extenso e ano)
{\ApprovalDate}%% Data de aprovação (dia, mês por extenso e ano)
{\DateAbreviated}%% Data de aprovação (forma abreviada; mestrado e doutorado)
{\linewidth}%% Largura de linha de assinatura (graduação e especialização)
%%%%%% Descrição do trabalho (padrão; alterar se necessário)
\DocumentTypeName\ apresentad\ifbool{Graduate}{a}{o} como requisito para obtenção do título de \StudentsTitlesList\ em \CourseName\ da \UTFPRName\ (\intl*{UTFPR}).
\end{ApprovalPage}
%%%% Opção 2: gerada a partir do Sistema Acadêmico ou da secretaria.
%%%% [Arg-1]: local (pasta) do PDF (./Pre-Textual/ por padrão).
%%%% {Arg-2}: nome do PDF (em ./Pre-Textual/; modelos em ./Pre-Textual/Extras/).
% \ApprovalPagePDF{doc-approval-page.pdf}