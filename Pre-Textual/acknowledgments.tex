%% Agradecimentos (elemento opcional)
%% Texto (pessoal) em que se fazem agradecimentos dirigidos àqueles que
%% contribuíram de maneira relevante à elaboração do trabalho.
\begin{Acknowledgments}%[Título Alternativo]%% Substitui o título padrão
Espaço destinado aos agradecimentos. Folha que contém manifestação de reconhecimento a pessoas e/ou instituições que realmente contribuíram com o(a) autor(a), devendo ser expressos de maneira simples.Não devem ser incluídas informações que nominem empresas ou instituições não nominadas no trabalho. Se o aluno recebeu bolsa de fomento à pesquisa, informar o nome completo da agência de fomento. Ex: Capes, CNPq, Fundação Araucária, UTFPR, etc. Incluir o número do projeto após a agência de fomento. Este item deve ser o último.

%%%% À agência de fomento (último): {Nome}; {Número/Código do Fomento}
% \FundingAgency{%
%   da \intldescr{CAPES} \textemdash\ Brasil (\intl*{CAPES})%% CAPES
% %   do \intl*{CNPq}, \intldescr{CNPq} \textemdash\ Brasil%% CNPq
% %   da Fundação Araucária \textemdash\ Brasil%% FA
% %   da \intl*{UTFPR}, \intldescr{UTFPR} \textemdash\ Brasil%% UTFPR
% }{%
%   \textemdash\ Código de Financiamento 001%% CAPES
% %   (\No\ de processo)%% CNPq
% %   (\No\ de edital, financiamento, processo ou projeto)%% FA
% %   (\No\ de edital, financiamento, processo ou projeto)%% UTFPR
% }
\end{Acknowledgments}