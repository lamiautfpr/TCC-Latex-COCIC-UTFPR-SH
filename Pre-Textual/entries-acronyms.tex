%%%% LISTA DE ABREVIATURAS E SIGLAS (ENTRADAS)
%%
%% Elemento opcional. Consiste na relação alfabética das abreviaturas e siglas
%% utilizadas no texto, seguidas das palavras ou expressões correspondentes
%% grafadas por extenso. Recomenda-se a elaboração de lista própria para cada
%% tipo.
%%
%% Observações:
%% - com subdivisões (\MakeAcronyms* em ./lamia-tcc-utfpr-sh.tex), os itens são
%%   ordenados pelo rótulo (label), mas conforme o tipo;
%% - sem subdivisões (\MakeAcronyms em ./lamia-tcc-utfpr-sh.tex), os itens são
%%   ordenados pelo rótulo (label), mas em uma única lista.

%% Abreviaturas e siglas
%% Definição:
%% \New<AcronymType>Entry{Label}{%% Usar somente letras latinas não acentuadas
%%   Term        = {...},%        % Obrigatório
%%   Description = {...},%        % Obrigatório
%%   Plural      = {...},%        % Opcional
%% }
%%%% Se a letra inicial do termo, da descrição ou do plural for acentuada, deve
%%%% ser colocada entre chaves, por exemplo, Description = {{á}rea}.
%% Impressão no texto e adição automática na lista:
%% +========================+==========================================+
%% | Comando                | Imprime                                  |
%% +========================+==========================================+
%% | \<acrtype>{Label}      | Termo                                    |
%% +------------------------+------------------------------------------+
%% | \<AcrType>{Label}      | Termo com letra inicial em maiúscula     |
%% +------------------------+------------------------------------------+
%% | \<acrtype>descr{Label} | Descrição                                |
%% +------------------------+------------------------------------------+
%% | \<AcrType>Descr{Label} | Descrição com letra inicial em maiúscula |
%% +------------------------+------------------------------------------+
%% | \<acrtype>pl{Label}    | Plural                                   |
%% +------------------------+------------------------------------------+
%% | \<AcrType>Pl{Label}    | Plural com letra inicial em maiúscula    |
%% +========================+==========================================+
%% Adição automática também no Índice Remissivo, usando um asterisco opcional:
%% +======================+=======================================+
%% | Comando              | Imprime                               |
%% +======================+=======================================+
%% | \<acrtype>*{Label}   | Termo                                 |
%% +----------------------+---------------------------------------+
%% | \<AcrType>*{Label}   | Termo com letra inicial em maiúscula  |
%% +----------------------+---------------------------------------+
%% | \<acrtype>pl*{Label} | Plural                                |
%% +----------------------+---------------------------------------+
%% | \<AcrType>Pl*{Label} | Plural com letra inicial em maiúscula |
%% +======================+=======================================+
%% Sendo:
%% +=============+===============+===========+===========+
%% | Tipo        | <AcronymType> | <acrtype> | <AcrType> |
%% +=============+===============+===========+===========+
%% | Abreviatura | Abbreviation  | abrv      | Abrv      |
%% +-------------+---------------+-----------+-----------+
%% | Sigla       | Initials      | intl      | Intl      |
%% +=============+===============+===========+===========+

%% Definições de abreviaturas
\NewAbbreviationEntry{art}{%
  Term        = {art.},%
  Description = {artigo},%
  Plural      = {arts.},%
}
\NewAbbreviationEntry{cap}{%
  Term        = {cap.},%
  Description = {capítulo},%
  Plural      = {caps.},%
}
\NewAbbreviationEntry{sec}{%
  Term        = {seç.},%
  Description = {seção},%
  Plural      = {seçs.},%
}

%% Definições de siglas
\NewInitialsEntry{ABNT}{%
  Term        = {ABNT},%
  Description = {Associação Brasileira de Normas Técnicas},%
}
\NewInitialsEntry{BMP}{%
  Term        = {BMP},%
  Description = {Mapa de Bits, do Inglês \ENLang*{Bitmap}},%
}
\NewInitialsEntry{CAPES}{%
  Term        = {CAPES},%
  Description = {Coordenação de Aperfeiçoamento de Pessoal de Nível Superior},%
}
\NewInitialsEntry{CNPq}{%
  Term        = {CNPq},%
  Description = {Conselho Nacional de Desenvolvimento Científico e Tecnológico},%
}
\NewInitialsEntry{CTAN}{%
  Term        = {CTAN},%
  Description = {\ENLang{Comprehensive \gly*{TeX} Archive Network}},%
}
\NewInitialsEntry{EPS}{%
  Term        = {EPS},%
  Description = {\ENLang{Encapsulated PostScript}},%
}
\NewInitialsEntry{GIF}{%
  Term        = {GIF},%
  Description = {Formato de Intercâmbio de Gráficos, do Inglês \ENLang*{Graphics Interchange Format}},%
}
\NewInitialsEntry{GIMP}{%
  Term        = {GIMP},%
  Description = {Programa de Manipulação de Imagem \intl*{GNU}, do Inglês \intl*{GNU} \ENLang*{Image Manipulation Program}},%
}
\NewInitialsEntry{GNU}{%
  Term        = {GNU},%
  Description = {GNU Não é Unix, do Inglês GNU \ENLang*{is Not Unix}},%
}
\NewInitialsEntry{JPEG}{%
  Term        = {JPEG},%
  Description = {\ENLang{Joint Photographic Experts Group}},%
}
\NewInitialsEntry{NBR}{%
  Term        = {NBR},%
  Description = {Norma Brasileira},%
}
\NewInitialsEntry{PDF}{%
  Term        = {PDF},%
  Description = {Formato de Documento Portátil, do Inglês \ENLang*{Portable Document Format}},%
}
\NewInitialsEntry{PNG}{%
  Term        = {PNG},%
  Description = {Gráficos Portáteis de Rede, do Inglês \ENLang*{Portable Network Graphics}},%
}
\NewInitialsEntry{PS}{%
  Term        = {PS},%
  Description = {\ENLang{PostScript}},%
}
\NewInitialsEntry{QR}{%
  Term        = {QR},%
  Description = {Resposta Rápida, do Inglês \ENLang*{Quick Response}},%
}
\NewInitialsEntry{TCC}{%
  Term        = {TCC},%
  Description = {Trabalho de Conclusão de Curso},%
}
\NewInitialsEntry{TUG}{%
  Term        = {TUG},%
  Description = {\ENLang{\TeX\ Users Group}},%
}
\NewInitialsEntry{UML}{%
  Term        = {UML},%
  Description = {Linguagem de Modelagem Unificada, do Inglês \ENLang*{Unified Modeling Language}},%
}
\NewInitialsEntry{URL}{%
  Term        = {URL},%
  Description = {Localizador Uniforme de Recursos, do Inglês \ENLang*{Uniform Resource Locator}},%
}
\NewInitialsEntry{UTFPR}{%
  Term        = {UTFPR},%
  Description = {Universidade Tecnológica Federal do Paraná},%
}
