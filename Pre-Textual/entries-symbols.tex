%%%% LISTA DE SÍMBOLOS (ENTRADAS)
%%
%% Elemento opcional. Conjunto de sinais que substituem o nome de uma coisa ou
%% de uma ação. Elaborada conforme a ordem apresentada no texto, com o devido
%% significado.
%%
%% Observações:
%% - com subdivisões (\MakeSymbols* em ./lamia-tcc-utfpr-sh.tex), os itens são
%%   ordenados pelo rótulo (label), mas conforme o tipo;
%% - sem subdivisões (\MakeSymbols em ./lamia-tcc-utfpr-sh.tex), os itens são
%%   ordenados pela ocorrência nos arquivos-fonte (impressão no texto).

%% Notações, sobrescritos, subscritos, letras gregas e letras latinas
%% Definição:
%% \New<SymbolType>Entry{Label}{%% Usar somente letras latinas não acentuadas
%%   Term        = {...},%       % Obrigatório
%%   Description = {...},%       % Obrigatório
%%   Unit        = {...},%       % Opcional
%%   Sort        = {...},%       % Opcional (para reordenar na lista)
%% }
%%%% Se a letra inicial da descrição for acentuada, deve ser colocada entre
%%%% chaves, por exemplo, Description = {{á}rea}.
%% Impressão no texto e adição automática na lista:
%% +========================+==========================================+
%% | Comando                | Imprime                                  |
%% +========================+==========================================+
%% | \<symtype>{Label}      | Termo                                    |
%% +------------------------+------------------------------------------+
%% | \<symtype>descr{Label} | Descrição                                |
%% +------------------------+------------------------------------------+
%% | \<SymType>Descr{Label} | Descrição com letra inicial em maiúscula |
%% +------------------------+------------------------------------------+
%% | \<symtype>unit{Label}  | Unidade                                  |
%% +========================+==========================================+
%% Adição automática também no Índice Remissivo, usando um asterisco opcional:
%% +====================+=========+
%% | Comando            | Imprime |
%% +====================+=========+
%% | \<symtype>*{Label} | Termo   |
%% +====================+=========+
%% Sendo:
%% +==============+==============+===========+===========+
%% | Tipo         | <SymbolType> | <symtype> | <SymType> |
%% +==============+==============+===========+===========+
%% | Notação      | Notation     | nttn      | Nttn      |
%% +--------------+--------------+-----------+-----------+
%% | Sobrescrito  | Superscript  | sprs      | Sprs      |
%% +--------------+--------------+-----------+-----------+
%% | Subscrito    | Subscript    | sbsc      | Sbsc      |
%% +--------------+--------------+-----------+-----------+
%% | Letra grega  | GreekLetter  | grkl      | GrkL      |
%% +--------------+--------------+-----------+-----------+
%% | Letra latina | LatinLetter  | ltnl      | LtnL      |
%% +==============+==============+===========+===========+
%%%% O comando \nttn{Label} possui um segundo argumento (opcional: [MrkSym]) que
%%%% armazena um símbolo no comando \MrkSym (\DottedCircle por padrão), no qual
%%%% se aplica a notação durante a impressão da mesma.

%% Definições de notações
\NewNotationEntry{averagea}{%
  Term        = {\overline{\MrkSym}},%
  Description = {média temporal},%
}
\NewNotationEntry{averageb}{%
  Term        = {\langle\MrkSym\rangle},%
  Description = {média na seção transversal},%
}
\NewNotationEntry{gradient}{%
  Term        = {\vec{\nabla}},%
  Description = {operador gradiente},%
}

%% Definições de sobrescritos
\NewSuperscriptEntry{minus}{%
  Term        = {^-},%
  Description = {passo de tempo anterior},%
}
\NewSuperscriptEntry{plus}{%
  Term        = {^+},%
  Description = {passo de tempo posterior},%
}
\NewSuperscriptEntry{zero}{%
  Term        = {^0},%
  Description = {valor inicial},%
}

%% Definições de subscritos
\NewSubscriptEntry{G}{%
  Term        = {_\mathrm{G}},%
  Description = {fase gasosa},%
}
\NewSubscriptEntry{L}{%
  Term        = {_\mathrm{L}},%
  Description = {fase líquida},%
}
\NewSubscriptEntry{S}{%
  Term        = {_\mathrm{S}},%
  Description = {fase sólida},%
}

%% Definições de letras gregas
\NewGreekLetterEntry{mu}{%
  Term        = {\mu},%
  Description = {viscosidade dinâmica},%
  Unit        = {kg/{(m{}\cdot{}s)}},%
}
\NewGreekLetterEntry{nu}{%
  Term        = {\nu},%
  Description = {viscosidade cinemática},%
  Unit        = {m^2/s},%
}
\NewGreekLetterEntry{pi}{%
  Term        = {\pi},%
  Description = {constante circular (Pi)},%
  Unit        = {rad},%
}
\NewGreekLetterEntry{rho}{%
  Term        = {\rho},%
  Description = {massa específica},%
  Unit        = {kg/m^3},%
}
\NewGreekLetterEntry{theta}{%
  Term        = {\theta},%
  Description = {inclinação},%
  Unit        = {\text{\textdegree}},%
}

%% Definições de letras latinas
\NewLatinLetterEntry{A}{%
  Term        = {A},%
  Description = {{á}rea},%
  Unit        = {m^2},%
}
\NewLatinLetterEntry{D}{%
  Term        = {D},%
  Description = {diâmetro},%
  Unit        = {m},%
}
\NewLatinLetterEntry{L}{%
  Term        = {L},%
  Description = {comprimento},%
  Unit        = {m},%
}
\NewLatinLetterEntry{R}{%
  Term        = {R},%
  Description = {raio},%
  Unit        = {m},%
}
\NewLatinLetterEntry{Re}{%
  Term        = {\mathrm{Re}},%
  Description = {número de Reynolds},%
}
\NewLatinLetterEntry{V}{%
  Term        = {V},%
  Description = {velocidade},%
  Unit        = {m/s},%
}
