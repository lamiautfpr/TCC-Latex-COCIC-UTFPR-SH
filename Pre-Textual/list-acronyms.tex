%%%% LISTA DE ABREVIATURAS E SIGLAS
%%
%% Elemento opcional. Consiste na relação alfabética das abreviaturas e siglas
%% utilizadas no texto, seguidas das palavras ou expressões correspondentes
%% grafadas por extenso. Recomenda-se a elaboração de lista própria para cada
%% tipo.

%% Lista de abreviaturas e siglas (inserir itens em ordem alfabética)
\begin{AcronymsList}%[Título Alternativo]%% Substitui o título padrão
\item[ABNT] Associação Brasileira de Normas Técnicas
\item[Art.] Artigo
\item[BMP] Mapa de Bits, do Inglês \ENLang*{Bitmap}
\item[Cap.] Capítulo
\item[CAPES] Coordenação de Aperfeiçoamento de Pessoal de Nível Superior
\item[CNPq] Conselho Nacional de Desenvolvimento Científico e Tecnológico
\item[CTAN] \ENLang{Comprehensive \TeX\ Archive Network}
\item[EPS] \ENLang{Encapsulated PostScript}
\item[GIF] Formato de Intercâmbio de Gráficos, do Inglês \ENLang*{Graphics Interchange Format}
\item[GIMP] Programa de Manipulação de Imagem GNU, do Inglês GNU \ENLang*{Image Manipulation Program}
\item[GNU] GNU Não é Unix, do Inglês GNU \ENLang*{is Not Unix}
\item[JPEG] \ENLang{Joint Photographic Experts Group}
\item[NBR] Norma Brasileira
\item[PDF] Formato de Documento Portátil, do Inglês \ENLang*{Portable Document Format}
\item[PNG] Gráficos Portáteis de Rede, do Inglês \ENLang*{Portable Network Graphics}
\item[PS] \ENLang{PostScript}
\item[QR] Resposta Rápida, do Inglês \ENLang*{Quick Response}
\item[Seç.] Seção
\item[TCC] Trabalho de Conclusão de Curso
\item[TUG] \ENLang{\TeX\ Users Group}
\item[UML] Linguagem de Modelagem Unificada, do Inglês \ENLang*{Unified Modeling Language}
\item[URL] Localizador Uniforme de Recursos, do Inglês \ENLang*{Uniform Resource Locator}
\item[UTFPR] Universidade Tecnológica Federal do Paraná
\end{AcronymsList}
