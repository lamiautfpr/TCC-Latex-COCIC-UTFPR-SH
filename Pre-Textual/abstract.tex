%% Resumo
%% Apresentação concisa dos pontos relevantes de um texto, fornecendo uma visão
%% rápida e clara do conteúdo e das conclusões do trabalho.
%% Ambiente {Abstract*}: insere a autorreferência do documento.
%%%% Estilo de fonte da chamada das palavras-chave (opcional)
% \KeywordsCallFormat{\bfseries}%% Texto normal por padrão
%%%% Palavras-chave (de 3 a 6): {Número}; {Em Português}; {In English}
\Keyword{1}{palavra-chave}{keyword}
\Keyword{2}{palavra-chave}{keyword}
\Keyword{3}{palavra-chave}{keyword}
% \Keyword{4}{palavra-chave}{keyword}

%%%% Em língua vernácula (idioma primário)
\begin{Abstract}[brazilian]%% Idioma (brazilian ou english)
O resumo deve ressaltar de forma sucinta o conteúdo do trabalho, incluindo justificativa, objetivos, metodologia, resultados e conclusão. Deve ser redigido em um único parágrafo, justificado, contendo de 150 até 500 palavras. Evitar incluir citações, fórmulas, equações e símbolos no resumo. A referência é elemento opcional em trabalhos acadêmicos, sendo que na UTFPR adotamos por não incluí-la nos resumos contidos nos próprios trabalhos. As palavras-chave e as keywords são grafadas em inicial minúscula quando não forem nome próprio ou nome científico e separados por ponto e vírgula.

\end{Abstract}

%%%% Em língua estrangeira (idioma secundário; para divulgação internacional)
\begin{Abstract}[english]%% Idioma (brazilian ou english)
Seguir o mesmo padrão do resumo, com a tradução do texto do resumo e referência, se houver, para a língua estrangeira (língua inglesa).
\end{Abstract}