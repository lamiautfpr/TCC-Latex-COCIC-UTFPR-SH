%% Epígrafe (elemento opcional)
%% Texto em que se apresenta uma citação, seguida de indicação de autoria,
%% relacionada com a matéria tratada no corpo do trabalho.
%%%% Opção 1: baseada na ABNT NBR 10520 (citações diretas curtas e longas).
%%%% Ambiente {Epigraph*}: remove o formato de citação direta longa.
% \begin{Epigraph}%% Argumentos (2):
% % [Deslocamento Vertical]%% Escala de comprimento a partir da margem superior
% % [Título]%% Não se aplica
% %%%%%% Epígrafes nos idiomas primário (texto) e original (nota de rodapé)
% %%%%%% [Arg-1]: idioma (brazilian ou english).
% %%%%%% {Arg-2}: autoria.
% %%%%%% {Arg-3}: texto.
% %%%%%% [Arg-4]: nota de rodapé.
% \Citation[brazilian]{\cite[tradução]{Einstein1921}}{%
%   Até onde as leis da matemática se referem à realidade, não são certas; e até onde são certas, não se referem à realidade.
% }[%
%   \Citation[english]{\cite{Einstein1921}}{%
%     As far as the laws of mathematics refer to reality, they are not certain; and as far as they are certain, they do not refer to reality.
%   }.
% ].
% \par%
% \Citation[brazilian]{\cite[p.~37, tradução]{Asimov1950}}{%
%   Primeira Lei: um robô não pode ferir um ser humano ou, por omissão, permitir que um ser humano sofra algum mal.
%   Segunda Lei: um robô deve obedecer às ordens que lhe sejam dadas por seres humanos, exceto nos casos em que tais ordens contrariem a Primeira Lei.
%   Terceira Lei: um robô deve proteger sua própria existência desde que tal proteção não entre em conflito com a Primeira ou Segunda Leis.
% }[%
%   \Citation[english]{\cite[37]{Asimov1950}}{%
%     First Law: a robot may not injure a human being or, through inaction, allow a human being to come to harm.
%     Second Law: a robot must obey the orders given it by human beings except where such orders would conflict with the First Law.
%     Third Law: a robot must protect its own existence as long as such protection does not conflict with the First or Second Laws.
%   }.
% ].
% \end{Epigraph}
%%%% Opção 2: baseada na classe de documento memoir.
% \begin{Epigraphs}%% Argumentos (2):
% [Deslocamento Vertical]%% Escala de comprimento a partir da margem superior
% [Título]%% Não se aplica
%%%%%% Epígrafes nos idiomas primário (texto) e original (nota de rodapé)
%%%%%% [Arg-1]: idioma (brazilian ou english).
%%%%%% {Arg-2}: autoria.
%%%%%% {Arg-3}: texto.
%%%%%% [Arg-4]: nota de rodapé.
% \QItem[brazilian]{\cite[tradução]{Einstein1921}}{%
%   Até onde as leis da matemática se referem à realidade, não são certas; e até onde são certas, não se referem à realidade.
% }[%
%   \Citation[english]{\cite{Einstein1921}}{%
%     As far as the laws of mathematics refer to reality, they are not certain; and as far as they are certain, they do not refer to reality.
%   }.
% ]
% \end{Epigraphs}