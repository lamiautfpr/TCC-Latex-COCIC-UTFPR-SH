%%%% GLOSSÁRIO
%%
%% Relação de palavras ou expressões técnicas de uso restrito, ou de sentido
%% obscuro, utilizadas no texto, acompanhadas das respectivas definições.

%% Glossário (inserir itens em ordem alfabética)
\begin{Glossary}%[\bfseries]%% Estilo de fonte do termo
\item[biber] substituto do Bib\TeX\ para usuários do Bib\LaTeX.
\item[Bib\LaTeX] reimplementação completa das facilidades bibliográficas fornecidas pelo \LaTeX.
\item[Bib\LaTeX-abnt] pacote que oferece um estilo Bib\LaTeX\ que atende as regras da ABNT\@.
\item[Bib\TeX] aplicativo de gerenciamento de referências para a formatação de listas de referências no \LaTeX.
\item[componente] outro exemplo de uma entrada secundária (componente), subentrada da primária chamada pai; trata-se de uma entrada irmã de outra também secundária chamada filho.
\item[dissertação] trabalho acadêmico desenvolvido no mestrado.
\item[equilíbrio da configuração] consistência entre os componentes.
\item[filho] exemplo de uma entrada secundária (filho), subentrada da primária chamada pai.
\item[\LaTeX] conjunto de macros para o processador de textos \TeX, utilizado amplamente para a produção de textos matemáticos e científicos devido à sua alta qualidade tipográfica.
\item[memoir] classe \LaTeX\ que permite a composição de poesia, ficção, obras de não ficção e matemáticas, como livros, relatórios, artigos ou manuscritos.
\item[pai] exemplo de entrada primária (pai) que possui subentradas ou entradas secundárias (filhos).
\item[tese] trabalho acadêmico desenvolvido no doutorado.
\item[\TeX] sistema de tipografia criado por Donald E. Knuth.
\item[\lamia-tcc-utfpr-sh] modelo \LaTeX\ que permite atender os requisitos das normas definidas pela UTFPR para elaboração de trabalhos acadêmicos.
\end{Glossary}
