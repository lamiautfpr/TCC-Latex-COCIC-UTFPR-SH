%%%% APÊNDICE (A)
%%
%% Texto ou documento elaborado pelo autor, de modo a complementar sua
%% argumentação, sem prejuízo da unidade nuclear do trabalho.

%% Locais (pastas) de ilustrações deste capítulo
\graphicspath{%
  {./Post-Textual/}%% Primário
%   {./Post-Textual/Illustrations/}%% Secundário (descomentar se houver)
}

\chapter{Título do Apêndice A}%
\label{chpt:apx-a}

Documentos auxiliares e/ou complementares, como legislações, estatutos, gráficos, tabelas, etc., podem ser apresentados na forma de apêndices, quando necessário.
Os apêndices, assim como os anexos, são enumerados com letras maiúsculas, por exemplo, \Cref{chpt:apx-a}.
Utilizam-se letras maiúsculas dobradas quando esgotadas as letras do alfabeto.

Apêndices complementam o texto principal do documento com informações para leitores com especial interesse no tema, devendo ser considerados leitura opcional, ou seja, o entendimento do texto principal do documento não deve exigir a leitura atenta dos apêndices.

Apêndices usualmente contemplam provas de teoremas, deduções de fórmulas matemáticas, diagramas esquemáticos, gráficos e trechos de código numérico.
Quanto a este último, um código numérico extenso não deve fazer parte do documento, mesmo como apêndice.
O ideal é disponibilizar o código numérico na Internet para os interessados em examiná-lo ou utilizá-lo, por exemplo, na plataforma \href{https://codeocean.com/}{Code Ocean\LinkIcon}, entre outras.

\section{Título de seção secundária do Apêndice A}%
\label{sect:apx-a2}

Exemplo de seção secundária de apêndice (\Cref{sect:apx-a2}).

\subsection{Título de seção terciária do Apêndice A}%
\label{ssect:apx-a3}

Exemplo de seção terciária de apêndice (\Cref{ssect:apx-a3}).

\subsubsection{Título de seção quaternária do Apêndice A}%
\label{sssect:apx-a4}

Exemplo de seção quaternária de apêndice (\Cref{sssect:apx-a4}).

\paragraph{Título de seção quinária do Apêndice A}%
\label{prgh:apx-a5}

Exemplo de seção quinária de apêndice (\Cref{prgh:apx-a5}).

\subparagraph{Título de parágrafo do \Cref{prgh:apx-a5}}%
\label{sprgh:apx-a6}

exemplo de parágrafo (divisão de seção quinária) de apêndice (\Cref{sprgh:apx-a6}).

\section{Ambientes matemáticos e atalhos úteis}%
\label{sect:math}

O \Cref{tfrm:math} apresenta os ambientes matemáticos e seus respectivos atalhos úteis em \gly*{TeX}/\gly*{LaTeX}.

\begin{tabframed}[!htbp]
\SetCaptionWidth{\textwidth}
\caption{Ambientes matemáticos e atalhos úteis}%
\label{tfrm:math}
\begin{tabularx}{\CaptionWidth}{?{}X*{3}{|>{\columncolor{shadecolor}}Y}?{}}%% CHKTEX 44
\toprule%
\multicolumn{1}{?{}Y|}{\textbf{Tipo}}                                           &
\multicolumn{1}{Y|}{\textbf{Fórmulas embutidas (no texto)}}                     &
\multicolumn{1}{Y|}{\textbf{Equações destacadas}}                               &
\multicolumn{1}{Y?{}}{\textbf{Equações destacadas e numeradas automaticamente}} \\
\midrule%
Ambiente                         &
\texttt{math}                    &
\texttt{displaymath}             &
\texttt{equation}\rlap{$^{(1)}$} \\
\midrule%
Atalho \gly*{LaTeX}                           &
\texttt{\textbackslash(\ldots\textbackslash)} &
\texttt{\textbackslash[\ldots\textbackslash]} &
{\textendash}                                 \\
\midrule%
Atalho \gly*{TeX}       &
\texttt{\$\ldots\$}     &
\texttt{\$\$\ldots\$\$} &
{\textendash}           \\
\bottomrule%
\end{tabularx}
\SourceOrNote*[]{\MathBF{^{(1)}} Versão com asterisco (\texttt{equation*}) suprime a numeração (pacote \gly*{LaTeX} \texttt{amsmath})}
\SourceOrNote{autoria própria (\YearNum)}
\end{tabframed}
