%%%% ANEXO (B)
%%
%% Texto ou documento não elaborado pelo autor, que serve de fundamentação,
%% comprovação e ilustração.

%% Locais (pastas) de ilustrações deste capítulo
\graphicspath{%
  {./Post-Textual/}%% Primário
  {./Post-Textual/Illustrations/}%% Secundário (descomentar se houver)
}

\chapter{Mapa com a Localização dos Campi da UTFPR}%
\label{chpt:anx-b}

A \Cref{fig:campi-map} apresenta um mapa com a localização dos 13 campi da \intl*{UTFPR}.

\begin{figure}[!htbp]
\SetCaptionWidth{0.7\textwidth}
\caption{Mapa com a localização dos campi da \intl*{UTFPR}}%
\label{fig:campi-map}
\savebox0{\includegraphics[width = \CaptionWidth]{fig-campi-map}}
\usebox0\llap{\raisebox{\ht0-\height}{\qrcode[height = 15mm]{https://www.utfpr.edu.br/campus}}}
\SourceOrNote{\textcite{UTFPR2017}}
\end{figure}

\section{Título de seção secundária do Anexo B}%
\label{sect:anx-b2}

Exemplo de seção secundária de anexo (\Cref{sect:anx-b2}).

\subsection{Título de seção terciária do Anexo B}%
\label{ssect:anx-b3}

Exemplo de seção terciária de anexo (\Cref{ssect:anx-b3}).

\subsubsection{Título de seção quaternária do Anexo B}%
\label{sssect:anx-b4}

Exemplo de seção quaternária de anexo (\Cref{sssect:anx-b4}).

\paragraph{Título de seção quinária do Anexo B}%
\label{prgh:anx-b5}

Exemplo de seção quinária de anexo (\Cref{prgh:anx-b5}).

\subparagraph{Título de parágrafo do \Cref{prgh:anx-b5}}%
\label{sprgh:anx-b6}

exemplo de parágrafo (divisão de seção quinária) de anexo (\Cref{sprgh:anx-b6}).
