\chapter{INTRODUÇÃO}\label{chp:INTRODUCAO} 
%Use referenciais sempre que necessário
Esta seção fornece uma introdução geral ao trabalho, destacando o tema e a delimitação do assunto tratado.

\section{Objetivo}\label{sec:OBJETIVOS}
%Esse texto fica como está

Expõem-se a seguir os objetivos geral e específicos que se pretende atingir com o trabalho.

\subsection{Geral}\label{sec:Geral}
%Descreva o objetivo do trabalho que é bem próximo a hipótese que você propos, veja os outros trabalhos de TCC para se inspirar

Descreve o objetivo geral do estudo, geralmente relacionado à hipótese ou propósito central do trabalho.

\subsection{Específicos}\label{sec:Especificos}
%Descreva aqui os objetivos específicos, veja os outros trabalhos de TCC para se inspirar
\begin{enumerate}
    \item Enumera objetivos específicos detalhados que serão alcançados no decorrer do trabalho;
    
    \item Objetivo especifico 2;
    
    \item Objetivo especifico 3;
    
    \item Objetivo especifico 4;
\end{enumerate}


\section{Contribuições do Trabalho}\label{sec:CONTRIBUICOES}

Descreve a principal contribuição do trabalho.

\begin{enumerate}
    \item Descreve contribuições pontuais do trabalho;
    
    \item Contribuição 2;
    
    \item Contribuição 3;
    
    \item Contribuição 4;
\end{enumerate}

\section{Justificativa}\label{sec:JUSTIFICATIVA}

Essa seção visa contextualizar a relevância do seu trabalho, mostrando por que ele é necessário e como isso pode contribuir para a resolução do problema encontrado.

\section{Delimitações do trabalho}\label{sec:DELIMITACOES}
%Você irá listar algumas delimitações utilizando uma estrutura de enumerate, veja os outros trabalhos de TCC para se inspirar
Enumera as delimitações do trabalho, indicando o que não será abordado ou comparado:

\begin{enumerate}
    \item Delimitação 1;
    \item Delimitação 2;
    \item Delimitação 3;
    \item Delimitação 4;
\end{enumerate}
