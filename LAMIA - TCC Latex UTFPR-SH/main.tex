%%This work may be distributed and/or modified under the conditions of the LaTeX Project Public License, either version 1.3 of this license or (at your option) any later version.
%--------------------------------------------------------------

% TEMPLATE PARA TRABALHO DE CONCLUSÃO DE CURSO
% Universidade Tecnológica Federal do Paraná - UTFPR
% Disponibilizado e mantido pelo LAMIA - Laboratório de Aprendizado de Máquina e Images Aplicados à Indústria
% lamia-sh@utfpr.edu.br
% https://www.lamia.sh.utfpr.edu.br/ - https://github.com/lamiautfpr - https://www.instagram.com/lamiautfpr/ - https://www.facebook.com/lamiautfpr2
% Campus Santa Helena
% Bacharelado em Ciência da Computação
% Customização da classe abnTeX2 (http://www.abntex.net.br/) para as normas da UTFPR - SH
% LaTeX:  abnTeX2   
% Projeto hospedado em: <https://github.com/lamiautfpr/TCC-Latex-COCIC-UTFPR-SH>
% Autor: Thiago França Naves - naves@utfpr.edu.br
% Colaboradores: João Ewerton Sousa, Paulo Vitor Souza

\pdfminorversion=6 % o pdf será gerado na versão 1.6
\documentclass{lamia-tcc-utfpr-sh}

%%%%%%%%%%%%%%%%%%%%%%%%%%%%%%%%%%%%%%%%%%%%%%%%%%%%%%%%%%%%
%P A C O T E S
%%%%%%%%%%%%%%%%%%%%%%%%%%%%%%%%%%%%%%%%%%%%%%%%%%%%%%%%%%%%
% Adicione aqui seus pacotes

\usepackage{lastpage} % cria uma ref LastPage para a última página do documento que pode ser utilizada através do \pageref

\usepackage{csquotes} %pacote com comandos para fazer citações de longos textos ou parágrafos de alguma referência

%pacote com todas as funções de bibliograficas modernas do latex, pesquise as variações do comando \cite disponíveis com o biblatex  

\usepackage[
  backend=biber,
  style=abnt,
  uniquename=mininit,
  giveninits,
  extrayear,
  repeattitles,
  backref=false,
  noslsn]{biblatex} %pacote com todas as funções de bibliograficas modernas do latex, pesquise as variações do comando \cite e \textcite disponíveis com o biblatex 

\addbibresource{elementos-postextuais/referencias.bib} %endereço do seu arquivo com as referências bibliográficas
\hypersetup{
    colorlinks=true,       		% Se não quiser cores no texto marcar como false
    linkcolor=black,          	   	% cor dos links do texto como o sumário
    citecolor=black,        		% cor das citações no texto
    filecolor=magenta,      		% cor de arquivos externos 
    urlcolor=black,}             	% cor de url no texto e nas referências bibliográficas
\DeclareUnicodeCharacter{0301}{\'{e}} % habilita a adição de caracteres unices nas referência bibliográficas
%%%%%%%%%%%%%%%%%%%%%%%%%%%%%%%%%%%%%%%%%%%%%%%%%%%%%%%%%%%%
%C O N F I G U R A Ç Õ E S A D I C I O N A I S
%%%%%%%%%%%%%%%%%%%%%%%%%%%%%%%%%%%%%%%%%%%%%%%%%%%%%%%%%%%%
\usepackage[titles]{tocloft} % pacote customizar a lista de figuras e tabelas
\renewcommand{\cftfigpresnum}{\figurename\enspace}
\renewcommand{\cftfigaftersnum}{ -}
\renewcommand{\cftfignumwidth}{2cm}

\renewcommand{\cfttabpresnum}{\tablename\enspace}
\renewcommand{\cfttabaftersnum}{ -}
\renewcommand{\cfttabnumwidth}{2cm}

%%%%%%%%%%%%%%%%%%%%%%%%%%%%%%%%%%%%%%%%%%%%%%%%%%%%%%%%%%%%
%I N I C I O  D O  D O C U M E N T O
%%%%%%%%%%%%%%%%%%%%%%%%%%%%%%%%%%%%%%%%%%%%%%%%%%%%%%%%%%%%
\begin{document}


% título da monografia é obrigatório
\title{\textbf{Principais Configurações na Integração de Visão Computacional e Aprendizagem Profunda: Algoritmos e Técnicas}}
% Nome do Curso é obrigatório

% autor é obrigatório; máximo de 3 autores (se TCC Empresa)
\author{Nome completo} {\input{elementos-pretextuais/agradecimentos}}
%\author{Nome completo aluno 2}{\input{elementos-pretextuais/agradecimentos}}
%\author{Nome completo aluno 3}
% orientador é obrigatório
\advisor[Prof.]{Prof. Me. ou Dr. Nome Completo}{}
% co-orientador é opcional
%\coadvisor[Prof.]{Nome do co-orientador,~M.Sc.}{}

% máximo de 3 integrantes da banca (orientador e co-orientador já são adicionados automaticamente)
\banca[Prof.]{Nome do participante banca 1,~D.Sc.}{}
\banca[Prof.]{Nome do participante banca 2,~D.Sc.}{}

\location{Santa~Helena}{PR}{Brasil}

% mês e ano de defesa
\date{Agosto}{2021}
\maketitle

% Hifenização de palavras que não estão no dicionário, coloque aquelas necessárias do seu texto
\hyphenation{%
	qua-dros-cha-ve
	Kat-sa-gge-los
}

\startdocument

%%%%%%%%%%%%%%%%%%%%%%%%%%%%%%%%%%%%%%%%%%%%%%%%%%%%%%%%%%%%
% D E D I C A T O R I A (opcional)
%%%%%%%%%%%%%%%%%%%%%%%%%%%%%%%%%%%%%%%%%%%%%%%%%%%%%%%%%%%% 
\makededicationpage

%%%%%%%%%%%%%%%%%%%%%%%%%%%%%%%%%%%%%%%%%%%%%%%%%%%%%%%%%%%%
% A G R A D E C I M E N T O S
%%%%%%%%%%%%%%%%%%%%%%%%%%%%%%%%%%%%%%%%%%%%%%%%%%%%%%%%%%%% 
\makethankspage

%%%%%%%%%%%%%%%%%%%%%%%%%%%%%%%%%%%%%%%%%%%%%%%%%%%%%%%%%%%%
% E P I G R A F E (opcional)
%%%%%%%%%%%%%%%%%%%%%%%%%%%%%%%%%%%%%%%%%%%%%%%%%%%%%%%%%%%% 
\makeepigraphpage

%%%%%%%%%%%%%%%%%%%%%%%%%%%%%%%%%%%%%%%%%%%%%%%%%%%%%%%%%%%%
% R E S U M O
%%%%%%%%%%%%%%%%%%%%%%%%%%%%%%%%%%%%%%%%%%%%%%%%%%%%%%%%%%%%
\begin{abstract}{
  % SOBRENOME, Prenome do Autor. Título do trabalho: subtítulo. Ano de defesa. \pageref{LastPage}f. (total de folhas). Trabalho de Conclusão de Curso (Bacharelado em Ciência da Computação) – Universidade Tecnológica Federal do Paraná. Santa Helena.

Elemento obrigatório, constituído de uma sequência de frases concisas e objetivas, fornecendo uma visão rápida e clara do conteúdo do estudo. O texto deverá conter no máximo 500 palavras e ser antecedido pela referência do estudo. Também, não deve conter citações. O resumo deve ser redigido em parágrafo único, espaçamento simples e seguido das palavras representativas do conteúdo do estudo, isto é, palavras-chave, em número de três a cinco, separadas entre si por ponto e finalizadas também por ponto. Usar o verbo na terceira pessoa do singular, com linguagem impessoal (pronome SE), bem como fazer uso, preferencialmente, da voz ativa.

Para definir as palavras-chave (e suas correspondentes em inglês no abstract) consultar em Termo tópico do Catálogo de Autoridades da Biblioteca Nacional, disponível em: \url{http://acervo.bn.br/sophia_web/index.html} [avaliar se essa informação procede para Ciência da Computação]

}
\bigskip
% Palavras-chave separadas por ponto
\palavraschave{Palavrachave1. Palavrachave2. Palavrachave3. Palavrachave4. Palavrachave5} \textcolor{red}{(separados entre si por ponto).}
\end{abstract}

%%%%%%%%%%%%%%%%%%%%%%%%%%%%%%%%%%%%%%%%%%%%%%%%%%%%%%%%%%%%
% A B S T R A C T
%%%%%%%%%%%%%%%%%%%%%%%%%%%%%%%%%%%%%%%%%%%%%%%%%%%%%%%%%%%%
\begin{englishabstract}{
  %% Resumo
%% Apresentação concisa dos pontos relevantes de um texto, fornecendo uma visão
%% rápida e clara do conteúdo e das conclusões do trabalho.
%% Ambiente {Abstract*}: insere a autorreferência do documento.
%%%% Estilo de fonte da chamada das palavras-chave (opcional)
% \KeywordsCallFormat{\bfseries}%% Texto normal por padrão
%%%% Palavras-chave (de 3 a 6): {Número}; {Em Português}; {In English}
\Keyword{1}{palavra-chave}{keyword}
\Keyword{2}{palavra-chave}{keyword}
\Keyword{3}{palavra-chave}{keyword}
% \Keyword{4}{palavra-chave}{keyword}

%%%% Em língua vernácula (idioma primário)
\begin{Abstract}[brazilian]%% Idioma (brazilian ou english)
O resumo deve ressaltar de forma sucinta o conteúdo do trabalho, incluindo justificativa, objetivos, metodologia, resultados e conclusão. Deve ser redigido em um único parágrafo, justificado, contendo de 150 até 500 palavras. Evitar incluir citações, fórmulas, equações e símbolos no resumo. A referência é elemento opcional em trabalhos acadêmicos, sendo que na UTFPR adotamos por não incluí-la nos resumos contidos nos próprios trabalhos. As palavras-chave e as keywords são grafadas em inicial minúscula quando não forem nome próprio ou nome científico e separados por ponto e vírgula.

\end{Abstract}

%%%% Em língua estrangeira (idioma secundário; para divulgação internacional)
\begin{Abstract}[english]%% Idioma (brazilian ou english)
Seguir o mesmo padrão do resumo, com a tradução do texto do resumo e referência, se houver, para a língua estrangeira (língua inglesa).
\end{Abstract}
}
\bigskip
% Palavras-chave separadas por ponto
\keywords{Keyword1. Keyword2. Keyword3. Keyword4. Keyword5} \textcolor{red}{(separados entre si por ponto).}
\end{englishabstract}

%%%%%%%%%%%%%%%%%%%%%%%%%%%%%%%%%%%%%%%%%%%%%%%%%%%%%%%%%%%%
% L I S T A S
%%%%%%%%%%%%%%%%%%%%%%%%%%%%%%%%%%%%%%%%%%%%%%%%%%%%%%%%%%%%
% Figuras (opcional)
\makefigurespage

% Tabelas (opcional)
\maketablespage

% Algoritmos (opcional)
%\makelistingspage

% Abreviaturas (devem estar em ordem alfabética) (opcional)
\makeabrevpage{\input{elementos-pretextuais/abreviaturas}}

% Símbolos (devem estar em ordem alfabética) (opcional)
\makesymbolspage{\input{elementos-pretextuais/simbolos}}

% Sumário 
\maketocpage

%%%%%%%%%%%%%%%%%%%%%%%%%%%%%%%%%%%%%%%%%%%%%%%%%%%%%%%%%%%%
% C O N T E Ú D O - T E X T O
%%%%%%%%%%%%%%%%%%%%%%%%%%%%%%%%%%%%%%%%%%%%%%%%%%%%%%%%%%%%
\startcontent
\chapter{INTRODUÇÃO}\label{chp:INTRODUCAO} 
%Use referenciais sempre que necessário
Esta seção fornece uma introdução geral ao trabalho, destacando o tema e a delimitação do assunto tratado.

\section{Objetivo}\label{sec:OBJETIVOS}
%Esse texto fica como está

Expõem-se a seguir os objetivos geral e específicos que se pretende atingir com o trabalho.

\subsection{Geral}\label{sec:Geral}
%Descreva o objetivo do trabalho que é bem próximo a hipótese que você propos, veja os outros trabalhos de TCC para se inspirar

Descreve o objetivo geral do estudo, geralmente relacionado à hipótese ou propósito central do trabalho.

\subsection{Específicos}\label{sec:Especificos}
%Descreva aqui os objetivos específicos, veja os outros trabalhos de TCC para se inspirar
\begin{enumerate}
    \item Enumera objetivos específicos detalhados que serão alcançados no decorrer do trabalho;
    
    \item Objetivo especifico 2;
    
    \item Objetivo especifico 3;
    
    \item Objetivo especifico 4;
\end{enumerate}


\section{Contribuições do Trabalho}\label{sec:CONTRIBUICOES}

Descreve a principal contribuição do trabalho.

\begin{enumerate}
    \item Descreve contribuições pontuais do trabalho;
    
    \item Contribuição 2;
    
    \item Contribuição 3;
    
    \item Contribuição 4;
\end{enumerate}

\section{Justificativa}\label{sec:JUSTIFICATIVA}

Essa seção visa contextualizar a relevância do seu trabalho, mostrando por que ele é necessário e como isso pode contribuir para a resolução do problema encontrado.

\section{Delimitações do trabalho}\label{sec:DELIMITACOES}
%Você irá listar algumas delimitações utilizando uma estrutura de enumerate, veja os outros trabalhos de TCC para se inspirar
Enumera as delimitações do trabalho, indicando o que não será abordado ou comparado:

\begin{enumerate}
    \item Delimitação 1;
    \item Delimitação 2;
    \item Delimitação 3;
    \item Delimitação 4;
\end{enumerate}

\chapter{REVISÃO DA LITERATURA}\label{chp:REVISAO}

O desenvolvimento do trabalho é composto por 3 seções: Revisão da Literatura (ou Referencial Teórico); Metodologia; e Análise dos Resultados, e pode conter outras além dessas. A revisão da literatura deve ser apresentada em forma de texto e seu conteúdo demonstra conhecimento da literatura científica sobre o tema do trabalho. O texto pode ser dividido, para fins didáticos, em subseções. Esta seção é permeada de autores, é o local em que há mais intertextualidade no trabalho. Assim, inclui basicamente citações indiretas (paráfrases) e diretas (curtas e longas). Aqui, o autor explicita a contribuição de outros campos do conhecimento que são envolvidos na pesquisa e outras pesquisas relacionadas ao tema, as conclusões que esses autores chegaram, o que é consenso, as discordâncias entre autores. 


\section{Intertextualidade}\label{sec:INTERTEXTUALIDADE}
No referencial teórico e em outras seções em que a intertextualidade é necessária, devem-se citar trabalhos clássicos, mas priorizar trabalhos dos últimos 10 anos. Podem-se usar artigos científicos, livros, TCCs, dissertações, teses, monografias e sites oficiais. Não são permitidos textos jornalísticos, Wikipédia e de blogues. É importante a utilização de referências em inglês no trabalho, livros e principalmente artigos de revista. O banco do IEEE é uma boa sugestão de fonte de pesquisa nessa língua. 

Devem-se seguir as normas da Associação Brasileira de Normas Técnicas (ABNT)  NBR 10520, Informação e documentação – Citações em documentos – Apresentação,  para fazer a intertextualidade por referenciação. 

\section{Estado da arte}\label{sec:ESTADOARTE}

No referencial teórico deve haver uma subseção para o estado da arte, em que se apresenta uma busca de anterioridade sobre o produto a ser desenvolvido, por exemplo, um software ou hardware, uma metodologia, bem como as publicações mais atuais e conceituadas sobre o tema do TCC. Assim, nessa seção, são contextualizados trabalhos anteriores parecidos ou relacionados ao aqui descrito. 

\section{Numeração das seções}\label{sec:NUMERAÇÃO}
Seguir a ABNT NBR 6024, Informação e documentação – Numeração progressiva das seções de um documento – Apresentação. As seções são formatadas como segue, e podem ir somente até a quaternária:

\begin{table}
	\caption{Numeração progressiva de seção e sua formatação, segundo a ABNT.}
	\label{tab:numeracaosecao}
	\begin{tabular}{|p{14.7cm}|} 
		\hline
		\textbf{\large 1 TÍTULO NÍVEL 1 OU SEÇÃO PRIMÁRIA OU TÍTULO DE CAPÍTULO (TODAS AS LETRAS DE CADA PALAVRA MAIÚSCULA, NEGRITO)}  \\ 
		\hline
		1.1 TÍTULO NÍVEL 2 OU SEÇÃO SECUNDÁRIA (TODAS AS LETRAS DE CADA PALAVRA MAIÚSCULA)~ ~                                   \\ 
		\hline
		1.1.1 Titulo Nível 3 ou Seção Terciária (Primeira Letra de Cada Palavra Maiúscula)~ ~                                   \\ 
		\hline
		1.1.1.1 Titulo nível 4 ou seção quaternária (somente letra da primeira palavra maiúscula)~ ~ ~ ~ ~ ~ ~~                 \\
		\hline
	\end{tabular}
	\newline \footnotesize \textbf{Fonte: Baseado em \cite{Nbr2012}.} 
\end{table}

\section{Equações e algoritimos com Latex}\label{sec:LATEX}

\subsection{Equações}\label{sec:Equacoes}
Referência: \url{http://en.wikibooks.org/wiki/LaTeX/Mathematics}

Também: \url{http://en.wikibooks.org/wiki/LaTeX/Advanced_Mathematics}

\begin{equation}
(x + y)^2 = x^2 + 2xy + y^2
\label{eq:Teorema1}
\end{equation}

Referência: \url{https://en.wikipedia.org/wiki/ID3_algorithm}

\begin{equation} \label{eq:DT3} 
\phi^{entropia}(X, y) = -\sum_{l=1}^{k} rac_{\bullet, yl} \times \log_{2} rac_{\bullet, yl}
\end{equation}

\subsection{Algoritmos}\label{sec:Algoritmos}
Referência: \url{http://en.wikibooks.org/wiki/LaTeX/Source_Code_Listings}

\codec{C}{alg:LABEL_CODE_1}{elementos-textuais/codigo-c.txt}

\codejava{Java}{alg:LABEL_CODE_2}{elementos-textuais/codigo-java.txt}

Referência \url{https://www.geeksforgeeks.org/genetic-algorithms/}

\begin{algorithm}
	\caption{Algoritmo Genetico:}
	\label{alg:algoritmogenetico}
	\begin{algorithmic}[1]
		\STATE $d \leftarrow$ Valor como critério de parada
		\STATE $IniciaPopulacao(P, t)$
		\STATE $Avaliacao(P, t)$
		\WHILE{$t < d$}
		\STATE $t \leftarrow t + 1$
		\STATE $SelecionaReprodutores(P, t)$
		\STATE $CruzaSelecionados(P, t)$
		\STATE $MutaResultantes(P, t)$
		\STATE $AvaliaResultantes(P, t)$
		\STATE $AtualizaPopulacao(P, t)$
		\ENDWHILE
	\end{algorithmic}
\end{algorithm}


\chapter{METODOLOGIA}\label{chp:METODOLOGIA}

Nesta seção, descreve-se como o trabalho foi desenvolvido, explicitando sucintamente a metodologia, os materiais e processos empregados para a execução do trabalho e como os objetivos serão alcançados. Esta seção responde às perguntas: Como será feita a pesquisa? Com o quê? Como será procedida a pesquisa? Visa a explicar de forma detalhada todas as ações desenvolvidas no percurso da pesquisa para que possa ser validada como científica. Então, esta seção descreve um método ou adapta uma metodologia preexistente. 

Para os trabalhos que envolvem pesquisas de campo, devem ser apresentados os instrumentos utilizados (questionário abertos, semiabertos, estruturados etc.) e a pertinência deles para o objeto de investigação proposto no trabalho, para que a pesquisa seja atestada como científica. Nesse caso, deve-se responder: Quais são os caminhos para se chegar aos objetivos propostos? Qual é o tipo de pesquisa? Qual é o universo da pesquisa? Será utilizada a amostragem? Quais são os instrumentos de coleta de dados?  Como foram construídos os instrumentos de pesquisa? Que forma é usada para a tabulação de dados? Como serão interpretados e analisados os dados e informações? 

 \section{Estrutura de uma trabalho acadêmico}\label{sec:ESTRUTURATRAB}
A organização de um trabalho acadêmico obedece a normas adotadas pela instituição (Figura \ref{fig:EstruturaTrab}). Tais normas garantem a organização do trabalho e guiam o autor.

\begin{figure}[htb]
	\centering
	\caption{Estrutura para elaboração de trabalhos acadêmicos.}
	\includegraphics[scale=0.6]{imagens/Estrutura-Trabalhos.png} 
	\newline \footnotesize \textbf{Fonte}: \cite{webLink}.
	\label{fig:EstruturaTrab}
\end{figure}

\section{Temática do TCC}\label{sec:TEMÁTICATCC}
Os temas tratados no TCC devem estar relacionados ao objeto de estudo da Ciência da Computação e estar inseridos em suas subáreas (Quadro 2) e abordar um problema do mundo real, a fim de propor soluções e melhorias. O TCC deve relacionar conhecimentos adquiridos em várias das disciplinas cursadas. Por exemplo, podem envolver algumas das seguintes áreas: inteligência artificial, redes, robótica, pesquisa operacional, grafos, arquitetura de \textit{hardware}, paradigmas de linguagens de programação, comunicação de dados, computação gráfica, matemática/estatística.  

\begin{table}[t]
	\caption{Áreas que compõem a Ciência da Computação de acordo com o CNPQ.}
	\label{tab:AreasCNPQ}
	\begin{tabular}{|l|l|}
		\hline
		\textbf{10300007} & \textbf{CIÊNCIA DA COMPUTAÇÃO}                     \\ \hline
		10301003          & TEORIA DA COMPUTAÇÃO                               \\ \hline
		10301011          & COMPUTABILIDADE E MODELOS DE COMPUTAÇÃO            \\ \hline
		10301020          & LINGUAGEM FORMAIS E AUTÔMATOS                      \\ \hline
		10301038          & ANÁLISE DE ALGORITMOS E COMPLEXIDADE DE COMPUTAÇÃO \\ \hline
		10301046          & LÓGICAS E SEMÂNTICA DE PROGRAMAS                   \\ \hline
		10302000          & MATEMÁTICA DA COMPUTAÇÃO                           \\ \hline
		10302018          & MATEMÁTICA SIMBÓLICA                               \\ \hline
		10302026          & MODELOS ANALÍTICOS E DE SIMULAÇÃO                  \\ \hline
		10303006          & METODOLOGIA E TÉCNICAS DA COMPUTAÇÃO               \\ \hline
		10303014          & LINGUAGENS DE PROGRAMAÇÃO                          \\ \hline
		10303022          & ENGENHARIA DE SOFTWARE                             \\ \hline
		10303030          & BANCO DE DADOS                                     \\ \hline
		10303049          & SISTEMAS DE INFORMAÇÃO                             \\ \hline
		10303057          & PROCESSAMENTO GRÁFICO (GRAPHICS)                   \\ \hline
		10304002          & SISTEMA DE COMPUTAÇÃO                              \\ \hline
		10304010          & HARDWARE                                           \\ \hline
		10304029          & ARQUITETURA DE SISTEMAS DE COMPUTAÇÃO              \\ \hline
		10304037          & SOFTWARE BÁSICO                                    \\ \hline
		10304045          & TELEINFORMÁTICA                                    \\ \hline
	\end{tabular}
	\newline \footnotesize \textbf{Fonte: \cite{Capes2018}.}
\end{table}

\section{Sugestões de formatos de TCC}\label{sec:SUGESTOESTCC}
Os TCCs do curso de Bacharelado em Ciência da Computação podem tratar de desenvolvimento de \textit{softwares} comerciais; desenvolvimento de \textit{softwares} científicos; desenvolvimento de \textit{softwares} educacionais; desenvolvimento de metodologias; revisão bibliográfica; e TCC empresa.[não limitar, modalizar]

\subsection{TCC de Desenvolvimento de Software}\label{sec:DesenvolvimentoSof}

Trabalho de conclusão de curso que apresenta o desenvolvimento de um software, desde seu planejamento até o teste prático. Segundo \textcite{Pressman2011}, o software pode ser comercial ou de aplicação, ou seja, um programa sob medida que soluciona uma necessidade específica de negócio, então, é desenvolvido com a finalidade de ser comercializado ou com interesses empresariais. As aplicações nessa área processam dados comerciais ou técnicos de uma forma que facilite as operações comerciais e as tomadas de decisões técnico-administrativas. Ainda, o software desenvolvido pode ser científico ou de engenharia, ou seja, um software que auxilia as aplicações científicas e é geralmente caracterizado por algoritmos de processamento de números. 

Por fim, há o desenvolvimento de software embutido ou embarcado. Trata-se de software próprio para um determinado hardware. O software embutido é usado para controlar produtos e sistemas para os mercados industriais e de consumo, e pode executar funções limitadas e específicas (por exemplo, controle do painel para fornos de micro-ondas) ou oferecer recursos funcionais significativos e capacidade de controle (por exemplo, funções digitais em automóveis, tal como controle de combustível, sistemas de freio) \cite{Pressman2011}. Nesse caso, o TCC pode ter como objetivo produzir tanto o \textit{software} como o hardware.

\subsection{TCC de Análise e Desenvolvimento de Metodologia}\label{sec:AnaliseDese}
O TCC de desenvolvimento de metodologia refere-se à proposta de alternativas ao modelos tradicionais de desenvolvimento de software, [metodologias de redes]. As metodologias devem acelerar a construção de soluções tecnológicas e têm por objetivo a melhoria contínua dos processos, trazendo avanços de comunicação e interação entre a equipe e os usuários, mais organização para o alcance de metas, diminuição de erros e retrabalhos, mais colaboração e, sobretudo, respostas rápidas às mudanças. Tudo isso favorece a geração de mais produtividade para os desenvolvedores, além de redução de custos e até mais satisfação com o trabalho. Novas maneiras de administrar as equipes de TI em projetos de desenvolvimento de software são geradas em função das metodologias ágeis, por exemplo, fazendo com que os usuários sejam participantes na construção das soluções \cite{Sommerville2011}. 

\subsection{TCC de Revisão Bibliográfica}\label{sec:RevisaoBib}
Conforme esclarece \cite{Boccato2006},  a pesquisa bibliográfica busca a resolução de um problema (hipótese) por meio de referenciais teóricos publicados. Analisa e discute as várias contribuições científicas existentes na área de estudo. Esse tipo de pesquisa traz subsídios para o conhecimento sobre o que foi pesquisado, como e sob que enfoque e perspectivas foi tratado o assunto apresentado na literatura científica. Para tanto, é de suma importância que o pesquisador realize um planejamento sistemático do processo de pesquisa, que compreenda desde a definição temática, passando pela construção lógica do trabalho até a decisão de sua forma de comunicação e divulgação.

Assim, um TCC de revisão bibliográfica resgata o estado da arte na área de estudo escolhida e traz conclusões baseadas na análise da literatura revisada.

\subsection{TCC Empresa}\label{sec:Empresa}
Envolve a criação de um produto e sua comercialização, com plano de negócio. É válido somente para empresas pré-encubadas na UTFPR-SH. Embora uma empresa geralmente tenha sócios, o TCC Empresa deve ser individual como os demais TCCs

\subsection{Ilustrações}\label{sec:Ilustracoes}
As ilustrações são um apoio para ajudar no esclarecimento do texto, de modo que apenas ilustrações pertinentes devem ser usadas. Todas elas devem obrigatoriamente estar citadas no corpo do texto, antes de aparecerem. 
Se o produto a ser desenvolvido for um software, o diagrama de casos de uso geral (Figura \ref{fig:CasoDeUso}) deve constar na seção de metodologia, assim como os diagramas de atividade (Figura \ref{fig:Atividades}). Trechos de código, que também entram como figura, devem ser apresentados em pseudocódigo. Diagramas de classes, se houver necessidade de inclusão, devem constar como apêndice. Anexos e apêndices também devem estar referenciados no texto. 

\begin{figure}[htb]
	\centering
	\caption{Exemplo de diagrama de caso de uso geral.}
	\includegraphics[scale=0.6]{imagens/DiagramaCasoDeUso.png} 
	\newline \footnotesize \textbf{Fonte}: (Própria, 2021).
	\label{fig:CasoDeUso}
\end{figure}

\begin{figure}[htb]
	\centering
	\caption{Exemplo de diagrama de atividade.}
	\includegraphics[scale=0.6]{imagens/DiagramaAtividades.png} 
	\newline \footnotesize \textbf{Fonte}: \cite{Ventura2018}.
	\label{fig:Atividades}
\end{figure}

\chapter{ANÁLISE DOS EXPERIMENTOS E RESULTADOS}\label{chp:AnaliseResultados}

%A seção foi renomeada e o texto atual não será utilizado

Nesta seção devem ser colocados os resultados obtidos, bem como as discussões acerca destes.
\chapter{CRONOGRAMA}\label{chp:CRONOGRAMA}
Em um cronograma, apresentado em forma de quadro, deve-se detalhar as atividades relacionadas à execução do projeto, bem como o tempo a ser dispendido em cada uma delas e sua época de realização

\pagebreak

%%%%%%%%%%%%%%%%%%%%%%%%%%%%%%%%%%%%%%%%%%%%%%%%%%%%%%%%%%%%
% B I B L I O G R A F I A
%%%%%%%%%%%%%%%%%%%%%%%%%%%%%%%%%%%%%%%%%%%%%%%%%%%%%%%%%%%%
% Retirar esta parte se o trabalho não tiver bibliografia
% Se não quiser que apareça no sumário, basta retirar o [] (deixar apenas \printbibliography)
\printbibliography[
heading=bibintoc,
title={REFERÊNCIAS}
]
\newpage


%%%%%%%%%%%%%%%%%%%%%%%%%%%%%%%%%%%%%%%%%%%%%%%%%%%%%%%%%%%%
% G L O S S A R I O (opcional, remova o comentário caso queira utilizar)
%%%%%%%%%%%%%%%%%%%%%%%%%%%%%%%%%%%%%%%%%%%%%%%%%%%%%%%%%%%%
%\makeglossarypage{\input{elementos-pretextuais/glossario}}

%%%%%%%%%%%%%%%%%%%%%%%%%%%%%%%%%%%%%%%%%%%%%%%%%%%%%%%%%%%%
% A N E X O (opcional)
%%%%%%%%%%%%%%%%%%%%%%%%%%%%%%%%%%%%%%%%%%%%%%%%%%%%%%%%%%%%
\annex
\input{elementos-postextuais/anexo1.tex}


\end{document}
