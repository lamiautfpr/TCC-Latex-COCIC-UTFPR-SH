\chapter{INTRODUÇÃO}\label{chp:INTRODUCAO}

Descrever objetivamente o problema focalizado, sua relevância no contexto da área inserida e sua importância específica para o avanço do conhecimento, ou seja, caracterização do problema e a justificativa do trabalho (sua utilidade, potencial inovador, importância para a sociedade – ver Quadro 2). Citações aqui somente se forem indispensáveis à apresentação do problema. Apresentar o assunto, mostrando a evolução da pesquisa na área pretendida. Justificar a proposição e sua inserção na área proposta.

Ao longo de todo o texto, deve-se utilizar a terceira pessoa do singular na elaboração do texto, mantendo-se a forma impessoal. Usar expressões como “Cabe ressaltar que...”, “Entende-se que...”, em vez de “Ressaltamos que...” , “Entendemos que...”. 


\section{OBJETIVOS}\label{sec:OBJETIVOS}
Expõem-se a seguir os objetivos geral e específicos que se pretende atingir com o trabalho . Ver mais exemplos no Anexo 1. 

\subsection{Geral}\label{sec:Geral}
Desenvolver um sistema de Internet of Things (IoT) automatizado que controle a entrada e saída de pessoas em áreas de acesso restrito, identifique suas funções institucionais por meio da tecnologia RFID, e registre o acesso via portas com trava ou catraca.


\subsection{Específicos}\label{sec:Especificos}
\begin{enumerate}
	\item Estudar conceitos básicos relativos a controle de acesso orientado a contextos; 
	
	\item Fazer uma proposta inicial de ambiente;
	
	\item Implementar um protótipo;
	
	\item Desenvolver um módulo Web para controle por parte do administrador, que permita monitorar os registros e gerenciar permissões
\end{enumerate}


\section{CONTRIBUIÇÕES DO TRABALHO}\label{sec:CONTRIBUICOES}
Explicitar as contribuições do trabalho para a sociedade, ou seja, sua pertinência. Por exemplo, no desenvolvimento de um sistema de controle com IoT, a contribuição é o ganho que um sistema desses pode trazer à área de organização escolar, uma vez que automatiza por completo as tarefas, junto com uso eficiente de TFID aliado a um sistema de controle funcional pela Web. Com essas características, o sistema apresenta melhorias em relação ao controle tradicional feito nesses ambientes e supera outras abordagens que não consideram as automatizações aqui propostas. [verificar enumeração]


\section{JUSTIFICATIVA}\label{sec:JUSTIFICATIVA}
A justificativa refere-se a por que é importante e válida a realização do trabalho. Trata-se de convencer o leitor de que o trabalho de pesquisa  apresenta contribuições específicas para a Ciência da Computação. Deve exaltar a importância da pesquisa e a relação de outras pesquisas sobre os mesmos assuntos. [adaptar texto para diferenciar do \ref{sec:CONTRIBUICOES}, colocar exemplo aqui e diretriz no comentário].


\section{DELIMITAÇÕES DO TRABALHO}\label{sec:DELIMITACOES}
Identificar e justificar aqui as delimitações do trabalho em relação a sua construção e objetivos a ser alcançados. Limitações podem estar presentes na: forma como os experimentos foram conduzidos, por falta de, por exemplo, equipamento específico, software ou recursos em geral; forma como a metodologia foi estabelecida, ignorando alguma etapa que por ventura existe, mas não será abordada devido ao viés do trabalho; ou em características gerais do trabalho, em que alguma limitação foi imposta para adequação à construção do trabalho e satisfação dos objetivos principais. [escopo, exemplo]
